\documentclass{beamer}

\usepackage{default}
\usepackage{graphicx}
\usepackage[utf8x]{inputenc}
\usepackage{amsmath}
\usepackage{amsfonts}
\usepackage{amssymb}
\usepackage{geometry}
%\usepackage{movie15}
\usepackage{media9}
\usepackage{subfigure}
\usepackage{textpos}
\usepackage[spanish]{babel}
\usepackage{pict2e}
\usepackage{epstopdf}
\newcommand{\gnuplotexe}{/opt/local/bin/gnuplot}
\usepackage[shell]{gnuplottex}
\usepackage{verbatim}
\usepackage{listings}
\lstset{language=C++%
,basicstyle=\tiny\ttfamily%
,backgroundcolor=\color{black!20}%
,breaklines=true%
,keywordstyle=\bfseries\color{magenta!90}%
,commentstyle=\itshape\color{green!40!black}%
,identifierstyle=\color{black}%
,stringstyle=\color{red!95!black}%
,numbers=left%
,numberstyle=\tiny%
,identifierstyle=\color{blue}%
%,title=\lstname%
}
%%%%%%%%%%%%%%%%%%%%%%%%%%%%%%%%%%%%%%%%%%%%%%%%%%%%%%%
%%%%%%%%%%%%%%%%%%%%%%%%%%%%%%%%%%%%%%%%%%%%%%%%%%%%%%%
%%%%%%%%%%%%%%%%%%%%%%%%%%%%%%%%%%%%%%%%%%%%%%%%%%%%%%%
\newcommand{\currentWeek}{00}
%%%%%%%%%%%%%%%%%%%%%%%%%%%%%%%%%%%%%%%%%%%%%%%%%%%%%%%
%%%%%%%%%%%%%%%%%%%%%%%%%%%%%%%%%%%%%%%%%%%%%%%%%%%%%%%
%%%%%%%%%%%%%%%%%%%%%%%%%%%%%%%%%%%%%%%%%%%%%%%%%%%%%%%
\usetheme{Madrid}

\titlegraphic{\begin{textblock}{3}(-0.5,-3.4)
     \includegraphics[height=2.0cm, width=2.0cm]{buap02}
   \end{textblock}
     \begin{textblock}{3}(12.5,-3.4)
     \includegraphics[height=2.0cm, width=2.0cm]{fcq}
   \end{textblock}}
	\title[BUAP-FCQ]{Desarollo de software científico para el cálculo especializado de entalpías de formación}
	\author[Édgar García Juárez]
{
\textbf{T  E  S  I  S}
}
\institute[]
{  
  	Para obtener el grado de Licenciado en Química\\[0.3cm]
  	Presenta: \\
 	 \textbf{Édgar García Juárez} \\[0.5cm]
 	 Director y Asesor:\\
	\textbf{Dr. Juan Manuel Solano Altamirano}\\ 
	\textbf{Dr. Julio Manuel Hernández Pérez}\\ 
}
	\date{\today}



\begin{document}
\setbeamertemplate{page number in head/foot}[totalframenumber]
\setbeamertemplate{navigation symbols}{}
\frame{
\titlepage
}
\frame{\tableofcontents}

%*************************************************************
\section{Introducción}
%*************************************************************
%++++++++++++++++++++++++++++++++++++++++++++++++++++++++
\begin{frame}
\frametitle{Química computacional}
\end{frame}




%...................................................
\begin{frame}[fragile]
\frametitle{Importancia de las funciones termodinámicas}
\end{frame}
%...................................................
%++++++++++++++++++++++++++++++++++++++++++++++++++++++++
%++++++++++++++++++++++++++++++++++++++++++++++++++++++++
%...................................................
\begin{frame}[fragile]
\frametitle{}

\end{frame}


%*************************************************************
\section{Objetivos}
%*************************************************************
\begin{frame}
\end{frame}

%*************************************************************
\section{Metodología}
%*************************************************************
\begin{frame}
\end{frame}

%*************************************************************
\section{Resultados}
%*************************************************************
\begin{frame}
\end{frame}

%*************************************************************
\section{Conclusiones}
%*************************************************************
\begin{frame}
\end{frame}

\end{document}
   
%%%%%%%%%%%%%%%%%%%%%%%%%%%%%%%%%%%%%%%%%%%%%%%%%%%%%%%%%%%%%%%%%%%%%%%

\begin{frame}[fragile]
\frametitle{Ejemplo}
\end{frame}
   
%%%%%%%%%%%%%%%%%%%%%%%%%%%%%%%%%%%%%%%%%%%%%%%%%%%%%%%%%%%%%%%%%%%%%%%

\begin{frame}[fragile]
\frametitle{Ejemplo}

\begin{lstlisting}

\end{lstlisting}
\end{frame}
